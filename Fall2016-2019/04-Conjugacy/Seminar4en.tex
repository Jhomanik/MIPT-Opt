\documentclass[12pt]{beamer}
\usepackage{../latex-sty/mypres}
\usepackage[utf8]{inputenc}
\usepackage[english]{babel}
\usepackage[T2A]{fontenc}

\expandafter\def\expandafter\insertshorttitle\expandafter{%
  \insertshorttitle\hfill%
  \insertframenumber\,/\,\inserttotalframenumber}
\title[Seminar 4]{Optimization Methods. \\
 Seminar 4. Conjugate sets. Farkas' lemma}
\author{Alexandr Katrutsa}
\institute{Moscow Institute of Physics and Technology,\\
Department of Control and Applied Mathematics} 
\date{\today}

\begin{document}
\begin{frame}
\maketitle
\end{frame}

\begin{frame}{Reminder}
\begin{itemize}
\item Interior and relative interior of convex set
\item Projection onto set
\item Separation of convex sets
\item Support hyperplane
\end{itemize}
\end{frame}

\begin{frame}{Conjugate set}
\begin{block}{Conjugate set}
Let $X^*$ be a conjugate (dual) set to the set $X \subseteq \bbR^n$ such that
\vspace{-4mm}
\[
X^* = \{ \bp \in \bbR^n | \langle \bp, \bx \rangle \geq -1, \; \forall \bx \in X \}.
\]
\end{block}

\begin{block}{Conjugate cone}
If $X \subseteq \bbR^n$ is a cone, then
\vspace{-4mm} 
\[
X^* = \{ \bp \in \bbR^n | \langle \bp, \bx \rangle \geq 0, \; \forall \bx \in X \}.
\]
\end{block}

\begin{block}{Conjugate subspace}
If $X$ is a linear subspace of $\bbR^n$, then 
\vspace{-4mm} 
\[
X^* = \{ \bp \in \bbR^n | \langle \bp, \bx \rangle = 0, \; \forall \bx \in X \}.
\]
\end{block}
\end{frame}

\begin{frame}{Claims about conjugate sets}
\begin{theorem}
Let $X$ be an arbitrary subset of $\bbR^n$. Then
\vspace{-4mm}
\[
X^{**} = \overline{\text{conv }(X \cup \{0\})}.
\] 
\end{theorem}

\begin{theorem}
Let $X$ be a closed convex set with zero. 
Then $X^{**} = X$.
\end{theorem}

\begin{theorem}
If $X_1 \subset X_2$, then $X^*_2 \subset X^*_1$.
\end{theorem}
\end{frame}

\begin{frame}{Examples}
Find conjugate sets for the following sets:
\begin{enumerate}
\item Nonnegative orthant: $\bbR^n_+$
\item Cone of positive semidefinite matrices: $\bS^n_+$
\item $\{ (x_1, x_2) | |x_1| \leq x_2 \}$
\item $\{ \bx \in \bbR^n | \| x \| \leq r \}$
\item $\{ (\bx, t) \in \bbR^{n+1} | \| x \| \leq t \}$
\end{enumerate}
\end{frame}

\begin{frame}{Farkas' lemma}
\scriptsize
\begin{lemma}[Farkas]
Assume $\bA \in \bbR^{m \times n}$ и $\mathbf{b} \in \bbR^m$. Then exactly one of the following system is feasible:
\vspace{-4mm}
\begin{equation*}
\begin{split}
1)& \; \bA\bx = \mathbf{b}, \; \bx \geq 0\\
2)& \; \bp^{\T}\bA \geq 0, \; \langle \bp, \mathbf{b} \rangle < 0
\end{split}
\end{equation*}
\end{lemma}

\begin{block}{Important corollary}
Assume $\bA \in \bbR^{m \times n}$ и $\mathbf{b} \in \bbR^m$. Then exactly one of the following systems is feasible:
\vspace{-4mm}
\begin{equation*}
\begin{split}
1)& \; \bA\bx \leq \mathbf{b}\\
2)& \; \bp^{\T}\bA = 0, \; \langle \bp, \mathbf{b} \rangle < 0, \; \bp \geq 0
\end{split}
\end{equation*}
\end{block}

\begin{block}{Application}
If the feasible set in linear programming problem is nonempty and objective function is bounded below, then the problem is feasible.
\end{block}

\end{frame}

\begin{frame}{Geometric interpretation}
\begin{block}{Farkas' lemma from geometric perspective}
\begin{itemize}
\item $\bA\bx = \mathbf{b}$ with $\bx \geq 0$ means that $\mathbf{b}$ lies in cone generated by the columns of matrix $\bA$
\item $\bp^{\T}\bA \geq 0, \; \langle \bp, \mathbf{b} \rangle < 0$ means that there exists separation hyperplane between vector $\mathbf{b}$ and cone generated by the columns of matrix $\bA$
\end{itemize}
\end{block}
\end{frame}

\begin{frame}{Recap}
\begin{itemize}
\item Conjugate sets
\item Properties of conjugate sets
\item Farkas' lemma
\end{itemize}

\end{frame}

\end{document}